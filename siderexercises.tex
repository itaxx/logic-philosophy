\documentclass[sloppy, journal, git, bytitle]{humapap}
% NOTE: This document makes use of the 'humapap' class. I'm still working on that class, but hope soon to include it in this repository.

\usepackage{fouriernc}
\theaffiliation{university college london}
\thetitle {logic for philosophy} % Fist page (no caps is fine)
\thehtitle{the header} % Title in the header (no caps is fine)
\mydescription{UCL reading group on Ted Sider's book} % Hovers over the title. Can be left blank (~)
\thejournal{the journal}
\theyear{2013}
\remote{http://github.com/msteenhagen/logic-philosophy/commit/}
\begin{document}
\documenttitle

\section{Chapter one}
\section{Chapter two}
\subsection{Disjunction and equivalence}
\paragraph{(a)} We need to show that for any wff $\varphi$ and $\chi$, 
and any PL-interpretation $\mathcal{I}$, 
V$\mathcal{_J}$($\varphi\vee\chi$)=1 iff either V$\mathcal{_J}$($\varphi$)=1 or V$\mathcal{_J}$($\chi$)=1. 
We must first notice that ($\varphi \vee \chi$) is just shorthand for ($\sim\varphi\rightarrow\chi$), 
so that what we need to show is that for any wff $\varphi$ and $\chi$, 
and any PL-interpretation $\mathcal{I}$, V$\mathcal{_J}$($\sim\varphi\rightarrow\chi$)=1 iff either V$\mathcal{_J}$($\varphi$)=1 or V$\mathcal{_J}$($\chi$)=1. 
This can be done by showing, first, that, 
if V$\mathcal{_J}$($\sim\varphi\rightarrow\chi$)=1 then V$\mathcal{_J}$($\varphi$)=1 or V$\mathcal{_J}$($\chi$)=1, 
and subsequently that, if V$\mathcal{_J}$($\varphi$)=1 or V$\mathcal{_J}$($\chi$)=1 then V$\mathcal{_J}$($\sim\varphi \rightarrow\chi$)=1.

Let us start by proving that if V$\mathcal{_J}$($\sim\varphi\rightarrow\chi$)=1, 
then either V$\mathcal{_J}$($\varphi$)=1 or V$\mathcal{_J}$($\chi$)=1. 
We first assume that V$\mathcal{_J}$($\sim\varphi\rightarrow\chi$)=1. 
Recall that we have defined the valuation function for $\rightarrow$ as follows: 
V$\mathcal{_J}$($\phi\rightarrow\psi$)=1 iff either V$\mathcal{_J}$($\phi$)=0 or V$\mathcal{_J}$($\psi$)=1. 
Given our assumption, we can therefore say that either V$\mathcal{_J}$($\sim\varphi$)=0 or V$\mathcal{_J}$($\chi$)=1. 
But then, given the definition for the valuation function for $\sim$, 
which is V$\mathcal{_J}$($\sim\phi$)=1 iff V$\mathcal{_J}$($\phi$)=0, 
we can say that either V$\mathcal{_J}$($\varphi$)=1 or V$\mathcal{_J}$($\chi$)=1. 
This is what we wanted to show.

The second stage of the proof requires us to prove that 
if V$\mathcal{_J}$($\varphi$)=1 or V$\mathcal{_J}$($\chi$)=1 then V$\mathcal{_J}$($\sim\varphi\rightarrow\chi$)=1. 
So, let us assume that either V$\mathcal{_J}$($\varphi$)=1 or V$\mathcal{_J}$($\chi$)=1. 
By the valuation function for $\sim$, 
we see how our assumption implies that either  V$\mathcal{_J}$($\sim\varphi$)=0 or V$\mathcal{_J}$($\chi$)=1. 
And so, subsequently, by the valuation function for $\rightarrow$, 
we can demonstrate that our assumption implies that V$\mathcal{_J}$($\sim\varphi\rightarrow\chi$)=1. $\square$

\paragraph{(b)} We need to show that for any wff $\varphi$ and $\chi$, and any PL-interpretation $\mathcal{I}$, $V\mathcal{_J}(\varphi\leftrightarrow\chi)=1$ iff $V\mathcal{_J}(\phi)$ = $V\mathcal{_J}(\chi)$. We begin by noting that $\phi\leftrightarrow\psi$ is shorthand for $(\phi\rightarrow\psi)\wedge(\psi\rightarrow\phi)$, and that the latter in its turn is shorthand for $(\sim(\phi\rightarrow\psi)\rightarrow\sim(\psi\rightarrow\phi))$.  
This means that what we need to show is that, for any wff $\varphi$ and $\chi$, and any PL-interpretation $\mathcal{I}$, $V\mathcal{_J}(\sim((\phi\rightarrow\chi)\rightarrow\sim(\chi\rightarrow\phi)))=1$ iff $V\mathcal{_J}(\phi)$ = $V\mathcal{_J}(\chi)$. This can be done by showing, first that, if $V\mathcal{_J}(\sim((\phi\rightarrow\chi)\rightarrow\sim(\chi\rightarrow\phi)))=1$ then $V\mathcal{_J}(\phi)$ = $V\mathcal{_J}(\chi)$, and, subsequently, that if $V\mathcal{_J}(\phi)$ = $V\mathcal{_J}(\chi)$, then $V\mathcal{_J}(\sim((\phi\rightarrow\chi)\rightarrow\sim(\chi\rightarrow\phi)))=1$. 

Let us start by proving that if $V\mathcal{_J}(\sim((\phi\rightarrow\chi)\rightarrow\sim(\chi\rightarrow\phi)))=1$ then $V\mathcal{_J}(\phi)$ = $V\mathcal{_J}(\chi)$. We proceed by reductio, and so start by assuming that $V\mathcal{_J}(\phi) \neq  V\mathcal{_J}(\chi)$. This allows us to assume that, say, $V\mathcal{_J}(\phi)=1$ and that $V\mathcal{_J}(\chi)=0$. Recall that we have defined the valuation function for $\rightarrow$ as follows: $V\mathcal{_J}(\phi\rightarrow\psi)=1$ iff either $V\mathcal{_J}(\phi)=0$ or $V\mathcal{_J}(\psi)=1$. The converse of this is that $V\mathcal{_J}(\phi\rightarrow\psi)=0$ iff $V\mathcal{_J}(\phi)=1$ and $V\mathcal{_J}(\psi)=0$. We see that our assumptions about the valuation interpretation for $\psi$ and $\chi$ imply that $V\mathcal{_J}(\phi\rightarrow\chi)=0$. We now introduce the further assumption that $V\mathcal{_J}(\sim((\phi\rightarrow\chi)\rightarrow\sim(\chi\rightarrow\phi)))=1$, which is just the antecedent of the conditional we are trying to prove. By the definition of the valuation function of $\sim$, we see that $V\mathcal{_J}((\phi\rightarrow\chi)\rightarrow\sim(\chi\rightarrow\phi))=0$. But if we again rely on our definition of the valuation function of $\rightarrow$, we can demonstrate that $V\mathcal{_J}(\phi\rightarrow\chi)=1$ (and that $V\mathcal{_J}(\sim(\chi\rightarrow\phi))=0$). But we have already demonstrated that $V\mathcal{_J}(\phi\rightarrow\chi)=0$, and so we have a contradiction. This allows us to reject the assumption that $V\mathcal{_J}(\phi) \neq  V\mathcal{_J}(\chi)$, and so we have shown that $V\mathcal{_J}(\sim((\phi\rightarrow\chi)\rightarrow\sim(\chi\rightarrow\phi)))=1$ then $V\mathcal{_J}(\phi)$ = $V\mathcal{_J}(\chi)$.

Now we still need to prove that if $V\mathcal{_J}(\phi)$ = $V\mathcal{_J}(\chi)$, then $V\mathcal{_J}(\sim((\phi\rightarrow\chi)\rightarrow\sim(\chi\rightarrow\phi)))=1$. We assume that $V\mathcal{_J}(\phi)$ = $V\mathcal{_J}(\chi)$, and so can assume that either $V\mathcal{_J}(\phi)=1$ and that $V\mathcal{_J}(\chi)=1$, or that $V\mathcal{_J}(\phi)=0$ and that $V\mathcal{_J}(\chi)=0$. We proceed in two straightforward stages.  

First, we demonstrate that if we assume $V\mathcal{_J}(\phi)=1$ and that $V\mathcal{_J}(\chi)=1$, it follows that $V\mathcal{_J}(\sim((\phi\rightarrow\chi)\rightarrow\sim(\chi\rightarrow\phi)))=1$. We assume that $V\mathcal{_J}(\phi)=1$ and that $V\mathcal{_J}(\chi)=1$. Using our definitions for the valuation function of $\rightarrow$ and $\sim$, we can say that $V\mathcal{_J}(\phi\rightarrow\chi)=1$ and that $V\mathcal{_J}(\sim(\chi\rightarrow\phi))=0$. But this implies that $V\mathcal{_J}((\phi\rightarrow\chi)\rightarrow\sim(\chi\rightarrow\phi))=0$, which in turn implies that $V\mathcal{_J}(\sim((\phi\rightarrow\chi)\rightarrow\sim(\chi\rightarrow\phi)))=1$. This is the desired result. 

Second, we demonstrate that if we assume $V\mathcal{_J}(\phi)=0$ and that $V\mathcal{_J}(\chi)=0$, it equally follows that $V\mathcal{_J}(\sim((\phi\rightarrow\chi)\rightarrow\sim(\chi\rightarrow\phi)))=1$. Again, the valuation function of $\rightarrow$ and $\sim$ allow us to say that $V\mathcal{_J}(\phi\rightarrow\chi)=1$ and that $V\mathcal{_J}(\sim(\chi\rightarrow\phi))=0$. As shown, this implies that $V\mathcal{_J}((\phi\rightarrow\chi)\rightarrow\sim(\chi\rightarrow\phi))=0$, which in turn implies that $V\mathcal{_J}(\sim((\phi\rightarrow\chi)\rightarrow\sim(\chi\rightarrow\phi)))=1$. $\square$

\section{Logical consequence}
\paragraph{(a)} We need to establish that $\vDash[P\wedge (Q\vee R)]\rightarrow[(P\wedge Q)\vee(P\wedge R)]$. 
We can prove this by reductio. 
So let $\mathcal{J}$ be any PL-interpretation, and let us assume that 
$V\mathcal{_J}((P\wedge (Q\vee R))\rightarrow((P\wedge Q)\vee(P\wedge R)))=0$. 
Given our definition of the valuation function $\rightarrow$, we can say that 
(\emph{i}) $V\mathcal{_J}(P\wedge (Q\vee R))=1$, and that
(\emph{ii}) $V\mathcal{_J}((P\wedge Q)\vee(P\wedge R))=0$.
Given our definitions of the valuation function $\wedge$ and $\vee$, we can say that 
(\emph{iii}) $\mathcal{J}(P)=1$ (from (\emph{i})), and that 
(\emph{iv}) $V\mathcal{_J}(P\wedge Q)=0$ (from (\emph{ii})), and so that 
(\emph{v}) $\mathcal{J}(P)=0$ (from (\emph{ii})). 
We have now derived a contradiction from our assumption (\emph{iii} \& \emph{v}), 
and so $V\mathcal{_J}((P\wedge (Q\vee R))\rightarrow((P\wedge Q)\vee(P\wedge R)))=1$.




\paragraph{(b)} 
\paragraph{(c)}  




%Bibliography: \standardbib just loads a regular bibliography. All files have been loaded in the preamble.
\standardbib
\end{document}